%----------------------------------------------------------------------------------------
%	SECTION TITLE
%----------------------------------------------------------------------------------------

\cvsection{工作经历}

%----------------------------------------------------------------------------------------
%	SECTION CONTENT
%----------------------------------------------------------------------------------------

\begin{cventries}

%------------------------------------------------
\cventry
{高级算法工程师} % Job title
{LiAuto 理想汽车} % Organization
{深圳} % Location
{} % Date(s)
{ % Description(s) of tasks/responsibilities
\begin{cvitems}
\item {大规模城市3D建模,预研模型超过大疆制图的航拍3D重建精度,部署部门航拍和绕拍的业务场景。}
\item {量产车数据采集建模;实时世界模型研发和端到端部署,动态资产仿真交互。}
\end{cvitems}
}

\cventry
{Research Intern} % Job title
{Synthesia} % Organization
{London, UK} % Location
{Jun. 2022 - Oct. 2022} % Date(s)
{ % Description(s) of tasks/responsibilities
\begin{cvitems}
\item {高精度虚拟人建模,支撑Synthesia人体动画摄影棚体表建模,重现发丝级的3D重建。}
\end{cvitems}
}

\cventry
{Research Intern} % Job title
{Facebook 脸书} % Organization
{Seattle, USA} % Location
{2021.06 - 2021.10} % Date(s)
{ % Description(s) of tasks/responsibilities
\begin{cvitems}
\item {对Oculus虚拟眼镜进行功能拓展,对扫描到的眼球做附带语义信息的3D补全,方便虚拟眼镜对人眼的实时视线追踪。}
\end{cvitems}
}
%------------------------------------------------

\cventry
{Prize Winner} % Job title
{Microsoft Research 微软研究院} % Organization
{Redmond, USA} % Location
{2016.04 - 2016.05} % Date(s)
{ % Description(s) of tasks/responsibilities
\begin{cvitems}
\item {Make multi-thread deep learning for CNTK, awarded as global 2$^\text{nd}$ prize in \href{https://www.microsoft.com/en-us/research/academic-program/microsoft-open-source-challenge/}{\color{awesome-skyblue}{\underline{Microsoft open source challenge}}}.} % Title
\end{cvitems}
}

\cventry
{Software Engineer} % Job title
{Google 谷歌} % Organization
{北京} % Location
{2015.04 - 2016.09} % Date(s)
{ % Description(s) of tasks/responsibilities
\begin{cvitems}
\item {An initial developer of \href{https://github.com/tiny-dnn/tiny-dnn}{\color{awesome-skyblue}{\underline{tiny-dnn}}}, which is the deep learning backend for OpenCV}.
\item {Contributed 3 OpenCV modules: \href{https://www.youtube.com/watch?v=Mc20rTYdXTE}{\color{awesome-skyblue}{\underline{3D multi-task learning}}}, \href{https://summerofcode.withgoogle.com/archive/2016/projects/4623962327744512}{\color{awesome-skyblue}{\underline{quantized deep learning}}} and \href{https://summerofcode.withgoogle.com/archive/2019/projects/6247164785721344}{\color{awesome-skyblue}\underline{super resolution}}.}
\end{cvitems}
}

%------------------------------------------------

\end{cventries}
